\documentclass[12pt,a4paper]{amsart}

\usepackage{amsmath}
\usepackage{graphicx}
\usepackage{amsrefs}
\usepackage{parskip}
\usepackage{bm}
\usepackage{amssymb}
\usepackage{mathrsfs}

\newcommand{\legendre}[1]{\mathscr{L}(#1)}
\newcommand{\pderiv}[3][]{%
  \frac{\partial^{#1} #2}{\partial #3^{#1}}%
}
\newcommand{\tridiag}[3]{%
  \begin{bmatrix}
    #1 & #2 & 0 & \cdots & 0 \\
    #3 & #1 & #2 & \ddots & \vdots \\
    0 & #3 & #1 & \ddots & 0 \\
    \vdots & \ddots & \ddots & \ddots & #2 \\
    0 & \cdots & 0 & #3 & #1
  \end{bmatrix}%
}

\newcommand{\Identity}[1]{\mathbb{I}_{#1}}
\setlength{\parskip}{0.8em}

\title{Project Plan: Bullet Pointed List}
\author{Benjamin Mason}

\begin{document}

\maketitle

\section*{TITLE}
\begin{itemize}
    \item Something to do with quantum billiard in it, maybe mention the nodal line conjecture.
\end{itemize}

\section*{INTRODUCTION}
\begin{itemize}
    \item Something similar to my intro in my Project Plan, basically providing a key summary of what reading the paper is going to achieve. 
    \item Mention some heuristics and popular culture things to bring in some relation and maybe some history.
    \item Somewhere I need to talk about the basics of dynamical systems. Maybe in the Mechanics section.
\end{itemize}

\section*{Chaotic Example: The Logistic Equation}
\begin{itemize}
    \item A good idea to start with a simple system that can quickly show chaotic behaviour.
    \item When key topics are discussed, they can be related to this example.
    \item Introduces the basic example of an iterative map, which will be useful when coming to Poincaré mappings.
    \item Easily reuse content from the Project Plan with some reformatting and applying feedback from Stefan.
\end{itemize}

\section*{Lagrangian and Hamiltonian Mechanics}
\begin{itemize}
    \item Need to start with basics of Lagrangian mechanics to get to Hamiltonian mechanics, to then get to billiard systems.
    \item Introduce Degrees of Freedom here.
    \item $\mathcal{L}(\bm{q}, \bm{\dot{q}}, t) = \mathcal{T}(\bm{q}, \bm{\dot{q}}) - \mathcal{V}(\bm{q}, t)$
    \item Maybe worth mentioning holonomic constraints, but unsure if necessary.
    \item \textbf{Key Theorem:} Hamilton's Variational Principle, used to derive Euler-Lagrange Equations.
    \item With a Lagrangian $\mathcal{L}(\bm{q}, \bm{\dot{q}}, t)$ and $\bm{\phi}(t) = (\phi_{1}(t)\ \cdots \ \phi_{f}(t))$ a trajectory in interval $t_1 \leq t \leq t_2$ and boundary values $\bm{\phi}(t_{1}) = \bm{a}, \bm{\phi}(t_{2}) = \bm{b}$. This orbit is such that the action $S[\bm{q}] = \int_{t_{1}}^{t_{2}} dt \mathcal{L}(\bm{q}, \bm{\dot{q}}, t)$ assumes an extremum.
    \item So here we can define the functional derivative, to then get to the EL equations.
    \item $\frac{\delta F[f]}{\delta f(x)} = \lim_{\epsilon \to 0} \frac{F[f + \epsilon \delta(x - x_0)] - F[f]}{\epsilon}$
    \item $\frac{\delta S[q]}{\delta q_{k}} = \pderiv{\mathcal{L}}{q_{k}} - \frac{d}{dt} \pderiv{\mathcal{L}}{\dot{q_{k}}} = 0$ CHECK THIS EQUATION
    \item Define the Hamiltonian through Legendre Transformation – need to learn more about it.
    \item Legendre Transformation: $\legendre{f}(x) = x \frac{df}{dx} - f(x) = zg(z) - f(g(z)) = \legendre{z}$ for $x = g(z)$ and $z = \frac{df}{dx}.$ THIS SEEMS TO BE 1D, MAY NEED TO LOOK AT FOR MULTIPLE VARIABLES
    \item $(\legendre{\mathcal{L}})(q, \dot{q}, t) = \dot{q}\frac{d\mathcal{L}}{d\dot{q}} - \mathcal{L} = \mathcal{H}(q, p, t)$
    \item Introduce phase space, constants of motion, trajectories, orbits, and conjugate momenta.
    \item Derive Hamilton's equations; I need to learn more about deriving these directly through the Hamiltonian.
    \item $\dot{q}_{k} = \pderiv{\mathcal{H}}{p_k}$, $\dot{p}_{k} = -\pderiv{\mathcal{H}}{q_k}$
    \item Could speak on symmetries, but likely not relevant.
    \item Need to introduce action and angle variables
    \item Need to introduce integrable and non-integrable systems.
\end{itemize}

\section*{Chaotic Properties}
\begin{itemize}
    \item Use this section to talk about the basics of chaotic motion
    \item Need to talk about the sensitivity of initial conditions and Lyapunov exponent
    \item Need to talk about trajectories not visiting the same place twice i.e. not periodic or quasi periodic.
    \item NEED MORE IN THIS SECTION
\end{itemize}

\section*{Geometry of Chaos}
\begin{itemize}
    \item Need to talk about tori.
    \item Need to introduce periodic and quasi-periodic motion.
    \item Need to talk about how constants of motions constrain the phase space surface, i.e. why integrable systems cannot display chaotic motion but integrable can
    \item Need to introduce commensurate and incommensurate frequencies
    \item Need to introduce KAM theorem (only basics)
    \item Need to introduce Poincare sections and Poincare mappings (important for billiards)
\end{itemize}

\section*{Quantum Mechanics}
\begin{itemize}
    \item Need to show the quantisation of the Hamiltonian
    \item Need to discuss 1D Schrödinger Equation, and infinite potential well in 1D
    \item Then go on to 2D Schrödinger Equation, and infinte potential wells in 2D
    \item Probably use a square as an example
    \item Need to make clear about eigenvalues and eigenvectors to then go onto the billiard things (like the numeric stuff)
    \item When in 2D, discuss difference between analytic and numeric solutions, and why integrable systems are seperable and non-integrable are not.
    \item $-\frac{1}{2} \frac{d^2 \psi(x)}{dx^2} + V(x) \psi(x) = E \psi(x)$
    \item $(-\frac{1}{2} \nabla^{2} + V)\psi(x, y) = E\psi(x, y)$
    \item $\Hat{H}\Psi = E\Psi$ for $\Hat{H} = \Hat{K} + \Hat{V}$
\end{itemize}

\section*{Classical Billiards}
\begin{itemize}
    \item Need to set up what classical billiard systems are i.e. just the definition I wrote down - relate it back to game of pool
    \item Need to mention about convex shaped billiards - maybe a good idea to research more types of billiard
    \item Relate Hamiltonians equations to the Poincaré Mapping i.e. explain why this  mapping captures all the information of the dynamics of the system.
    \item Give general argument for deriving mapping for convex billiard
    \item Then use circle example as an integrable system, and cardioid as a non-integrable system
    \item Maybe useful to use Bill2D to get some diagrams in there, or there are many good diagrams available.
\end{itemize}

\section*{Quantum Billiards and Chaos}
\begin{itemize}
    \item Need to talk about quantum chaos is the quantum mechanics of classically chaotic systems.
    \item Talk about specific quantum chaos phenomena, like quantum scarring
    \item Relate the potential well examples to billiard systems
    \item Breifly highlight the nodal structure of wavefunctions to lead into the next section
\end{itemize}

\section*{Nodal Line Conjecture}
\begin{itemize}
    \item Need to talk about the nodal line conjecture
    \item Basically take the key details from the paper on this
    \item Set up the need for Q-BIDs
\end{itemize}

\section*{Numerical Methods for solving the Schrödinger Equation}
\begin{itemize}
    \item Deriver the numerical method that Q-BIDs uses.
    \item Taylor Expansion leads to finite-difference method
    \item In 1D; $\pderiv[2]{f}{x} \approx \frac{f_{i+1} - 2f_{i} + f_{i-1}}{h^{2}}$; 2D; $\nabla^{2} = L \approx D \oplus D = D \otimes \Identity{N} + \Identity{N} \otimes D$ for $D$ the discretised differential operator in 1D
    \item $D = \tridiag{-2}{1}{1}$
    \item Need to talk about the limitations of the numerical methods
\end{itemize}

\section*{Q-BIDs}
\begin{itemize}
    \item Discuss why I created it
    \item Need to talk about basic features
    \item May want to do a basic example of the pixel art to solving the equation
    \item May want code snippets (i do not know if necessary)
\end{itemize}

\section*{Results}

\section*{Conclusion}
\end{document}
